\documentstyle[11pt]{article}  
%\documentstyle{article}  
%\pagestyle{empty}  
%\parindent = 0pt  
%\parskip= .20in

\begin{document}

\title{g4.prim format version 2.4\\
for Fukui Renderer DAWN}
\author{Satoshi Tanaka}
\date{February 5, 1998}
\maketitle


%%%%%%%%%%%%%%%%%%%%%%%%%%%%%%%%
\section{Introduction}   
%%%%%%%%%%%%%%%%%%%%%%%%%%%%%%%%

This document explains the ``g4.prim format'' which is able to describe  
visualizable 3D (three-dimensional) scenes   
transferred from GEANT4 with help of its visualization manager 
and Fukui Renderer driver (DAWN driver).  
In the following, we use the term ^^ ^^ 3D data'' meaning  
a data to construct a 3D scene or a hint information for the construction,
e.g., shape, color, body-coordinate definition, and bounding box of
the whole scene.
On the other hand, the term ^^ ^^ g4.prim-format data'' means  
a set of 3D data plus several rendering commands to control DAWN.

DAWN receives the g4.prim-format data from GEANT4  
via the TCP/IP socket of the default port number 40701,  
or via the named pipe by specifying -G and -g options, respectively:  

\begin{verbatim}
     % dawn -G     (invoking DAWN with socket mode)
     % dawn -g     (invoking DAWN with named pipe mode)
\end{verbatim}

\noindent
GEANT4 invokes DAWN with -G option by default,  
and with -g option if the environmental variable  
``G4DAWN\_NAMED\_PIPE'' is set to 1.  
How GEANT4 sends the g4.prim-format data to DAWN is demonstrated in 
the test programs, ``\verb+g4test_inet.cc+'' and ``\verb+g4test_unix.cc+'', 
included in the DAWN package.
G4.prim-format data sent from GEANT4 are automatically saved  
to a file with the name ``g4.prim" in current directory of DAWN.   

G4.prim-format data can also be visualized  
with the stand alone (off-line) use of DAWN, giving a g4.prim-format file name  
from a command line as an argument.

\begin{verbatim}
     % dawn g4.prim-format-filename
\end{verbatim}

The g4.prim format is designed with the following philosophy:  

\begin{enumerate}
\item  The format should be suitable to describe 3D data of GEANT4.
       For this purpose,  
       (i) it should support most of shapes defined in GEANT4  
       as built-in primitives, and   
       (ii) it should be able to describe both the CSG and B-Rep data.  
\item  Attributes such as color should be described with the way of  
       state machine in order to avoid repetitive resetting of them.  
\item The format should be able to describe rendering commands for
      DAWN as well as 3D data,  
      but the rendering commands should be clearly distinguishable  
      from 3D data.  
\item The format should be suitable for effective data transfer  
      via local or wide area network. For this purpose,   
      it should be compact and be able to be translated into VRML format
      easily.
\end{enumerate}



%%%%%%%%%%%%%%%%%%%%%%%%%%%%%%%%%%%%%%%%
\section{3D Data and rendering commands}   
%%%%%%%%%%%%%%%%%%%%%%%%%%%%%%%%%%%%%%%%

Each 3D data begins with a character '/' (slash).  
For example, a $1\times{}1\times{}1$ box is described as:   

\begin{verbatim}
     /Box  0.5 0.5 0.5
\end{verbatim}

\noindent
where the 0.5's are half lengths of edges.
On the other hand,  
each rendering command for DAWN begins with a character '!' 
(exclamation mark).  
For example,  

\begin{verbatim}
     !DrawAll
\end{verbatim}
\noindent
is a command to request DAWN to flush drawing.  
Each 3D data and rendering command need not
begin at the first column of the line, i.e., 
indentation is allowed.

%%%%%%%%%%%%%%%%%%%%%%%%%%%%%%%%%%%%%%%%
\section{Comment lines and blank lines}   
%%%%%%%%%%%%%%%%%%%%%%%%%%%%%%%%%%%%%%%%

A Line beginning with a character '\verb+#+' is regarded as a comment line.  
The character '\verb+#+' must be put at the first column.
(Indentation of comment lines are not allowed.)
The first line of the whole data should be a comment line as  
^^ ^^ \verb+##G4.PRIM-FORMAT-2.4+''. 
We call this line the ^^ ^^ header-comment line'' below.  

Blank lines are acceptable except for the very first line of the whole date. 

%%%%%%%%%%%%%%%%%%%%%%%%%%%%%%%%%%%%%%
\section{Global structure of g4.prim format}   
%%%%%%%%%%%%%%%%%%%%%%%%%%%%%%%%%%%%%%

The g4.prim format consists of three blocks:  
(1) preamble (2) modeling block, and  
(3) terminating block

%-----------------------------
\subsection{Preamble}   
%-----------------------------

The preamble is of the following form:  

\begin{verbatim}
##G4.PRIM-FORMAT-2.4
/BoundingBox  xmin ymin zmin xmax ymax zmax   
!SetCamera  
!OpenDevice
\end{verbatim}

\noindent
The first line of a g4.prim-format data must be the header-comment line 
as ^^ ^^ \verb+##G4.PRIM-FORMAT-2.4+''.
The line, \verb+/BoundingBox ... +  
defines a bounding box which determines extension of the described 3D scene.  
Real numbers \verb+xmin+, \verb+ymin+ and \verb+zmin+ are minimum  
x, y, and z world coordinates of the 3D scene,  
while \verb+xmax+, \verb+ymax+ and \verb+zmax+ are maximum.  
DAWN uses this information for automatic camera positioning,  
automatic drawing of coordinate axes, etc.  
\vspace{.20in}

\noindent  
{\bf{}Example:}  

\begin{verbatim}
##G4.PRIM-FORMAT-2.4
/BoundingBox -1.0 -2.5 -5.0 1.0 2.5 5.0
!SetCamera
!OpenDevice
\end{verbatim}

%-----------------------------
\subsection{Modeling block}  
%-----------------------------

The modeling block begins with a line \ \verb+!BeginModeling+,   
and ends with a line                  \ \verb+!EndModeling+.
Three kinds of 3D data are described between them.

\noindent  
\begin{enumerate}
\item 3D primitives, i.e, shapes and polylines
\item Current body coordinates used to locate 3D primitives
\item Current attributes to be assigned to 3D primitives
\item Markers to be set to arbitrary 3D positions
\end{enumerate}

\noindent
The available 3D primitives, attributes, and markers  
are listed later together with a detailed description of  
their formats.

The current body coordinates are defined by the following two 
lines:

\begin{verbatim}
/Origin      Ox  Oy  Oz
/BaseVector  e1.x  e1.y  e1.z  e2.x  e2.y  e2.z  
\end{verbatim}

\noindent
where (\verb+Ox+, \verb+Oy+, \verb+Oz+) is the origin of   
the current body coordinates.  
Orientation of the current body coordinates is defined with  
a set of base vectors, i.e.,  
x-directional base vector (\verb+e1.x+, \verb+e1.y+, \verb+e1.z+)  
and y-directional base vector (\verb+e2.x+, \verb+e2.y+, \verb+e2.z+).
Z-directional base vector is defined as  
(\verb+e1.x+, \verb+e1.y+, \verb+e1.z+) $\times$  
(\verb+e2.x+, \verb+e2.y+, \verb+e2.z+).  
The base vectors need not be normalized.  
The origin and the base vectors are expressed in terms of the world
coordinates.  
(There is no concept of hierarchical coordinate transformation.)
The default current body coordinates, which need not be
explicitly described are:

\begin{verbatim}
/Origin      0.0 0.0 0.0  
/BaseVector  1.0 0.0 0.0  0.0 1.0 0.0   
\end{verbatim}

\noindent 
In other words, the default current body coordinates are identical 
with the world coordinates.

Attributes are able to be assigned to 3D primitives.  
The most important one is color: \verb+/ColorRGB R G B+.
For example, a red box is described by putting a line
\verb+/ColorRGB 1.0 0.0 0.0+ before a line \verb+/Box ...+.  
Each of R (red), G (green), and B (blue) components of color  
should take a value between 0 and 1.  
The default color which need not be explicitly  
described is white, which is equivalent to \verb+/ColorRGB 1.0 1.0 1.0+.  
Once current body coordinates or current attributes are set,  
they remain effective until   
they are explicitly described later again with other values.
\vspace{.20in}

\noindent  
{\bf{}Example: Red $1\times{}4\times{}9$ box centered at position (1, 2, 3)}

\begin{verbatim}
!BeginModeling
/Origin   1  2  3  
/ColorRGB 1  0  0  
/Box 0.5   2.0   4.5  
!EndModeling
\end{verbatim}


%-----------------------------
\subsection{Terminating block}
%-----------------------------
The terminating block describes a few rendering commands as follows.

\begin{verbatim}
!DrawAll
!CloseDevice
\end{verbatim}

\noindent
These commands request DAWN to flush drawing and close the current  
visualizing device.

%-----------------------------
\subsection{A Real example of complete visualizable file}
%-----------------------------

The following is a visualizable example of a   
red $1\times{}4\times{}9$ box centered at position   
(1,2,3) of the world coordinate.

\begin{verbatim}
##G4.PRIM-FORMAT-2.4
#####################################
# Red 1x4x9 box centered at (1,2,3) #
#####################################
/BoundingBox  0.5  0.0  -1.5   1.5   4.0   7.5
!SetCamera
!OpenDevice
!BeginModeling
/Origin   1.0  2.0  3.0
/ColorRGB 1.0  0.0  0.0  
/Box 0.5   2.0   4.5  
!EndModeling
!DrawAll
!CloseDevice  
\end{verbatim}


%%%%%%%%%%%%%%%%%%%%%%%%%%%%%%%%%%%
\section{Formats of 3D primitives}  
%%%%%%%%%%%%%%%%%%%%%%%%%%%%%%%%%%%

In this section, we explain formats of available 3D primitives,  
i.e., shapes and polylines, one by one.
Note that current body coordinates are used in the description.
For example, the x-axis means the one in the current body coordinates.
All angles are described in units of radians.


%------------------
\subsection{/Box}
%------------------

\verb+/Box+ describes a box with edges parallel to the x, y, and z axes.
Its center locates at the origin.
\verb+/Box+ corresponds to class G4Box of GEANT4.
\vspace{.20in}

\begin{tabular}{|c|l|}
\hline%---------------------------------------
Format & \verb+/Box dx  dy  dz+\\
\hline%---------------------------------------
\verb+dx+     & half length of x-directional edges\\
\hline%---------------------------------------
\verb+dy+     & half length of y-directional edges\\
\hline%---------------------------------------
\verb+dz+     & half length of z-directional edges\\
\hline%---------------------------------------
\end{tabular}


%%%%%%%%%%%%%%%%%%%%%%%%%%%%%%%%%%%%%%%%%%%%%%%%%%%%%%%%%%%%%%%%%%%%%%%
%%------------------
%\subsection{/Brep \ldots\ \ /EndBrep}  
%%------------------
%\verb+/Brep+ \ldots{} \verb+/EndBrep+ describes a general polyhedron  
%composed of a set of 3D polygons.  
%It is equivalent to \verb+/Polyhedron+ \ldots{} \verb+/EndPolyhedron+,
%but its format is different.
%It is recommended to use \verb+/Polyhedron+ \ldots{} \verb+/EndPolyhedron+.
%Concave polygons are available but not recommended.  
%Lists of vertices and
%facets are described between \verb+/Brep+ and \verb+/EndBrep+.  
%The following is the format for a polyhedron with  
%$N$ vertices and $M$ facets.
%\vspace{.20in}
%
%\begin{tabular}{|c|l|}
%\hline%---------------------------------------
%Format & \verb+/Brep+\\
%       & \verb+BeginVertex+\\
%       & \verb+v1  x1 y1 z1+\\
%       & \verb+v2  x2 y2 z2+\\
%       & \ldots\ldots    \\
%       & \verb+vN  xN yN zN+\\
%       & \verb+EndVertex+\\
%       & \verb+BeginFacet+\\
%       & \verb+f1_v1 f1_v2 f1_v3 +\ldots{};\\
%       & \verb+f2_v1 f2_v2 f2_v3 +\ldots{};\\
%       & \ldots\ldots    \\
%       & \verb+fM_v1 fM_v2 fM_v3 +\ldots{};\\
%       & \verb+EndFacet+\\
%       & \verb+/EndBrep+\\
%\hline%----------------------------------------------------
%\verb+BeginVertex ... EndVertex+  & sub-block to describe vertex positions\\
%\hline%----------------------------------------------------
%\verb+vi+   & integer vertex label of the i-th vertex\\
%            & of the polyhedron ($1 \leq \verb+i+ \leq N$) \\
%\hline%----------------------------------------------------
%\verb+(xi, yi, zi)+ & 3D position of the i-th vertex \\
%                    & of the polyhedron ($1 \leq \verb+i+ \leq N$) \\
%\hline%----------------------------------------------------
%\verb+BeginFacet ... EndFacet+  & sub-block to describe facets\\
%\hline%----------------------------------------------------
%\verb+fa_vj+         & the j-th vertex label of the a-th facet, \\
%                     & whose absolute value must coincide with \\
%		     & one of the vertex labels  
%		       \{\verb+v1+, \verb+v2+, \ldots, \verb+vN+\}\\
%\hline%----------------------------------------------------
%\verb+fa_v1 fa_v2 fa_v3 ... ;+ & the a-th facet expressed by connecting \\
%                   & vertex labels in counter-clock-wise order\\
%                   & viewing from its front side.\\
%                   & Edge \verb+fa_vi+---\verb+fa_vj+ is regarded as \\
%                   & an invisible edge if \verb+fa_vj+ is negative. \\
%		   & At least three vertices must be described.\\
%\hline%----------------------------------------------------
%\end{tabular}
%\vspace{.20in}
%
%The following is a sample description of a   
%red $1\times{}4\times{}9$ box.   
%\vspace{.20in}
%
%\noindent  
%{\bf{}Example: 1$\times{}4\times{}9$ box}
%
%\begin{verbatim}
%##G4.PRIM-FORMAT-2.4
%######################################
%# 1x4x9 box with /Brep ... /EndBrep ##
%######################################
%
%/Brep
%BeginVertex
%1 -0.500000000 -2.000000000 -4.500000000
%2 0.500000000 -2.000000000 -4.500000000
%3 -0.500000000 2.000000000 -4.500000000
%4 0.500000000 2.000000000 -4.500000000
%5 -0.500000000 -2.000000000 4.500000000
%6 0.500000000 -2.000000000 4.500000000
%7 -0.500000000 2.000000000 4.500000000
%8 0.500000000 2.000000000 4.500000000
%EndVertex
%BeginFacet
%3 4 2 1 ;
%1 2 6 5 ;
%2 4 8 6 ;
%4 3 7 8 ;
%3 1 5 7 ;
%5 6 8 7 ;
%EndFacet
%/EndBrep
%\end{verbatim}
%
%
%An invisible edge is expressed by assigning a negative 
%integer vertex label to its ending point.
%For example, ``1 -2 6 5 ;'' describes a facet
%1$\rightarrow$2$\rightarrow$6$\rightarrow$5$\rightarrow$1,
%in which the edge connecting vertices 1 and 2 is invisible.
%
%%%%%%%%%%%%%%%%%%%%%%%%%%%%%%%%%%%%%%%%%%%%%%%%%%%%%%%%%%%%%%%%%%%%


%------------------
\subsection{/Column}
%------------------
\verb+/Column+ describes a solid cylinder.  
Its height extends along the z axis.
The     top    circle is on plane z = +dz,   
and the bottom circle    on plane z = -dz.
Centers of the top and bottom circles are   
(0, 0, +dz) and  (0, 0, -dz), respectively.
\verb+/Column+ is equivalent to \verb+/Tubs+   
with full azimuthal angle ($2\pi$) and zero minimum radius.  
\vspace{.20in}

\begin{tabular}{|c|l|}
\hline%---------------------------------------
Format & \verb+/Column  r dz +\\
\hline%---------------------------------------
\verb+r+   & radius of the top and bottom circles\\
\hline%---------------------------------------
\verb+dz+  & half height along the z axis\\
\hline%---------------------------------------
\end{tabular}


%------------------
\subsection{/Cons}
%------------------
\verb+/Cons+ describes a tube or its segment in azimuthal angle  
with varying minimum (inside) and maximum (outside) radii.
It is a cone (segment) with its upper part cut away.
Its height extends along the z axis.
The     top    facet is on plane z = +dz,   
and the bottom facet    on plane z = -dz.
Centers of the top and bottom facets are   
(0, 0, +dz) and  (0, 0, -dz), respectively.
\verb+/Cons+ corresponds to class G4Cons of GEANT4.
\vspace{.20in}

\begin{tabular}{|c|l|}
\hline%---------------------------------------
Format & \verb+/Cons rmin1 rmax1 rmin2 rmax2 dz sphi dphi+\\
\hline%---------------------------------------
\verb+rmin1+  & minimum (inside) radius at z = -\verb+dz+\\
\hline%---------------------------------------
\verb+rmax1+  & maximum (outside) radius at z = -\verb+dz+\\
\hline%---------------------------------------
\verb+rmin2+  & minimum (inside)  radius at z = +\verb+dz+\\
\hline%---------------------------------------
\verb+rmax2+  & maximum (outside) radius at z = +\verb+dz+\\
\hline%---------------------------------------
\verb+dz+     & half height along the z axis\\
\hline%---------------------------------------
\verb+sphi+  &  starting azimuthal angle, $[-2\pi, +2\pi]$ \\
\hline%---------------------------------------
\verb+dphi+  &  extension of azimuthal angle, \verb+dphi+ = $[0, 2\pi]$, \\
             &  \verb+sphi+ + \verb+dphi+ = $[-2\pi, +2\pi$] \\
\hline%---------------------------------------
\end{tabular}


%----------------------------
\subsection{/Parallelepiped}  
%----------------------------
\verb+/Parallelepiped+ describes a parallelepiped,  
which is the following skewed box.
\begin{enumerate}
\item The top and bottom facets are identical parallelograms.
\item The top parallelogram is on plane     z = +dz  
      and the bottom parallelogram on plane z = -dz.
\item The center locates at the origin.
\item A line joining the centers of the top and bottom parallelograms is
      skewed by angles $\theta$ (polar angle) and $\phi$ (azimuthal angle).
\item Two edges of the top (or bottom) parallelogram are parallel  
      to the x axis.   
      Their length is $2\times$ \verb+dx+, and
      distance between them, i.e., height of the parallelogram,  
      is $2\times$ \verb+dy+.
\item The top (or bottom) parallelogram skews by angle $\alpha$ in the   
      x direction.
      That is, $\alpha$ is the angle formed the y axis and
      a line joining middle points of the two x-directional edges   
      mentioned above.
\end{enumerate}
\verb+/Parallelepiped+ corresponds to class G4Para of GEANT4.
\vspace{.20in}


\begin{tabular}{|c|l|}
\hline%---------------------------------------
Format & \verb+/Parallelepiped  dx dy dz+ \\
       & \verb+                 tanAlpha+\\
       & \verb+                 tanTheta_cosPhi+\\
       & \verb+                 tanTheta_sinPhi+\\
\hline%---------------------------------------
\verb+dx+     & half length of edges parallel to the x axis\\
\hline%---------------------------------------
\verb+dy+     & half height of the top and bottom parallelograms\\
              & along the y axis \\
\hline%---------------------------------------
\verb+dz+     & half height of this parallelepiped along the z axis\\
\hline%---------------------------------------
\verb+tanAlpha+  & $\tan(\alpha)$\\
                 & \ \ $\alpha$: angle expressing skew of the top and bottom\\   
                 & \ \ parallelograms in the x direction.\\
\hline%---------------------------------------
\verb+tanTheta_cosPhi+  &  $\tan(\theta)\times\cos(\phi)$ \\
                   & \ \ Polar angle $\theta$ and azimuthal angle $\phi$\\
		   & \ \ describes skewness of the line joining centers\\
                   & \ \ of the top and bottom parallelograms\\
\hline%---------------------------------------
\verb+tanTheta_sinPhi+  &  $\tan(\theta)\times\sin(\phi)$\\
\hline%---------------------------------------
\end{tabular}


%----------------------------
\subsection{/PolyCone}  
%----------------------------
\verb+/PolyCone+ describes a rotational body around the z axis.
The shape of \mbox{\verb+/PolyCone+} is piled-up \verb+/Cons+'s  
in the z direction.
\verb+/PolyCone+ corresponds to class G4BREPSolidPCone of GEANT4.
But \verb+/PolyCone+ is not really used in GEANT4 visualization.
\verb+/Polyhedron ... /EndPolyhedron+ is used instead.   
\vspace{.20in}

\begin{tabular}{|c|l|}
\hline%---------------------------------------
Format & \verb+/PolyCone  sphi  dphi  nz           + \\
       & \verb+           z[nz]  rmin[nz]  rmax[nz]+ \\
       & (A[n] abbreviates a list of n real numbers) \\
\hline%---------------------------------------
\verb+sphi+  & starting azimuthal angle, $[-2\pi, +2\pi$] \\
\hline%---------------------------------------
\verb+dphi+  &  extension of azimuthal angle, \verb+dphi+ = $[0, 2\pi]$\\
             &  and \verb+sphi+ + \verb+dphi+ = $[-2\pi, +2\pi$] \\
\hline%---------------------------------------
\verb+nz+ & number of given z coordinates to define piled-up
            \verb+/Cons+'s:\\
          & \verb+nz+ is equal to the number of the piled-up
	    \verb+/Cons+'s\\
	  & plus one.\\
\hline%---------------------------------------
\verb+z[i]+  & z coordinate of the top plane of the i-th \verb+/Cons+.\\  
             & The \verb+z[0]+ defines the bottom plane of the whole shape. \\
             & i = 0,1,2, ..., \verb+nz+ -1.\\
\hline%---------------------------------------
\verb+rmin[i]+  & minimum (inside)  radius at z = z[i] \\
\hline%---------------------------------------
\verb+rmax[i]+  & maximum (outside) radius at z = z[i] \\
\hline%---------------------------------------
\end{tabular}


%----------------------------
\subsection{/PolyGon}  
%----------------------------
\verb+/PolyGon+ is similar to \verb+/PolyCone+.
The difference is that its side is not a curved surface, but 
its side is really a set of explicitly given numbers of 3D polygons. 
\verb+/PolyGon+ corresponds to class G4BREPSolidPolyhedra of GEANT4.
But \verb+/PolyGon+ is not really used in GEANT4 visualization.   
\verb+/Polyhedron ... /EndPolyhedron+ is used instead.   
\vspace{.20in}

\begin{tabular}{|c|l|}
\hline%---------------------------------------
Format & \verb+/PolyGon   sphi  dphi  nsides  nz+ \\
       & \verb+           z[nz]  rmin[nz]  rmax[nz]+ \\
       & (A[n] abbreviates a list of n real numbers) \\
\hline%---------------------------------------
\verb+nsides+   & accuracy of expressing curved side surfaces\\
\hline%---------------------------------------
others & same as \verb+/PolyCone+ \\
\hline%---------------------------------------
\end{tabular}

%-------------------------------------------
\subsection{/Polyhedron \ldots\ \ /EndPolyhedron}  
%--------------------------------------------

\verb+/Polyhedron+ \ldots{} \verb+/EndPolyhedron+
describes a general polyhedron composed of a set of 3D polygons.  
%%%%%%%%%%%%%%%%%%%%%%%%%%%%%%%%%%%%%%%%%%%%%%%%%%%%%%%%%%%%
%It is equivalent to \verb+/Brep+ \ldots{} \verb+/EndBrep+,
%but its format is different.
%It is recommended to use \verb+/Polyhedron+ \ldots{} \verb+/EndPolyhedron+.
%%%%%%%%%%%%%%%%%%%%%%%%%%%%%%%%%%%%%%%%%%%%%%%%%%%%%%%%%%%%
Concave polygons are available but not recommended.  
Lists of vertices and
facets are described between \verb+/Polyhedron+ and \verb+/EndPolyhedron+.  
The following is the format for a polyhedron with  
$N$ vertices and $M$ facets.
\vspace{.20in}

\begin{tabular}{|c|l|}
\hline%---------------------------------------
Format & \verb+/Polyhedron+\\
       & \verb+/Vertex  x1 y1 z1+\\
       & \verb+/Vertex  x2 y2 z2+\\
       & \ldots\ldots    \\
       & \verb+/Vertex  xN yN zN+\\
       & \verb+/Facet  f1_v1 f1_v2 f1_v3 +\ldots{}\\
       & \verb+/Facet  f2_v1 f2_v2 f2_v3 +\ldots{}\\
       & \ldots\ldots    \\
       & \verb+/Facet  fM_v1 fM_v2 fM_v3 +\ldots{}\\
       & \verb+/EndPolyhedron+\\
\hline%----------------------------------------------------
\verb+(xi, yi, zi)+ & 3D position of the i-th vertex \\
                    & of the polyhedron ($1 \leq \verb+i+ \leq N$). \\
                    & The integer label ^^ i'' is assigned to \\
		    & vertices incrementally, beginning from 1.\\
\hline%----------------------------------------------------
\verb+fa_vj+         & the j-th vertex label of the a-th facet, \\
                     & whose absolute value must coincide with \\
		     & one of the vertex labels  
		       \{$1, 2, 3, \ldots, N$\}.\\
\hline%----------------------------------------------------
\verb+fa_v1 fa_v2 fa_v3 ... + & the a-th facet expressed by connecting \\
                   & vertex labels in counter-clock-wise order\\
                   & viewing from its front side.\\
                   & Edge \verb+fa_vi+---\verb+fa_vj+ is regarded as \\
                   & an invisible edge if \verb+fa_vj+ is negative. \\
		   & At least three vertices must be described.\\
\hline%----------------------------------------------------
\end{tabular}
\vspace{.20in}


Note that positive-integer labels are incrementally and automatically
assinged to vertices given by  \verb+/Vertex  ...+,   
beginning from 1, not 0.

An invisible edge is expressed by assigning a negative 
integer vertex label to its ending point.
For example, ``/Facet 1 -2 6 5 ;'' describes a facet
1$\rightarrow$2$\rightarrow$6$\rightarrow$5$\rightarrow$1,
in which the edge connecting vertices 1 and 2 is invisible.

Vertex data (\verb+/Vertex ...+) and facet data (\verb+/Facet  ...+)
can be described at any position with any order
between \verb+/Polyhedron+ \ldots{} \verb+/EndPolyhedron+, 
but it is recommended that all vertex data are described beforehand.  

The following is a sample description of a   
red $1\times{}4\times{}9$ box using
\verb+/Polyhedron+ \ldots{}\verb+/EndPolyhedron+. 
\vspace{.20in}

\noindent  
{\bf{}Example: 1$\times{}4\times{}9$ box}

\begin{verbatim}
##G4.PRIM-FORMAT-2.4
##################################################
# 1x4x9 box with /polyhedron ... /EndPolyhedron ##
##################################################

/BoundingBox -0.5 -2.0 -4.5  0.5  2.0  4.5
!SetCamera
!OpenDevice 
!BeginModeling

/ColorRGB  1.0 0.0 0.0

/Polyhedron
/Vertex -0.500000000 -2.000000000 -4.500000000
/Vertex  0.500000000 -2.000000000 -4.500000000
/Vertex -0.500000000 2.000000000 -4.500000000
/Vertex  0.500000000 2.000000000 -4.500000000
/Vertex -0.500000000 -2.000000000 4.500000000
/Vertex  0.500000000 -2.000000000 4.500000000
/Vertex -0.500000000 2.000000000 4.500000000
/Vertex  0.500000000 2.000000000 4.500000000
/Facet 3 4 2 1 
/Facet 1 2 6 5 
/Facet 2 4 8 6 
/Facet 4 3 7 8 
/Facet 3 1 5 7 
/Facet 5 6 8 7 
/EndPolyhedron


!EndModeling
!DrawAll 
!CloseDevice 
\end{verbatim}



%-------------------------------------------
\subsection{/Polyline \ldots\ \ /EndPolyline}  
%--------------------------------------------
\verb+/Polyline ... /EndPolyline+ describes a polyline, i.e.,  
a set of successive line segments.  
A List of vertices is described between
\verb+/Polyline+ and \verb+/EndPolyline+.
The following is the format of describing a polyline with
$N$ vertices, i.e., $N-1$ line segments.
\vspace{.20in}

\begin{tabular}{|c|l|}
\hline%---------------------------------------
Format & \verb+/Polyline+\\
       & \verb+/PLVertex x1 y1 z1+\\
       & \verb+/PLVertex x2 y2 z2+\\
       & \ldots\ldots    \\
       & \verb+/PLVertex xN yN zN+\\
       & \verb+/EndPolyline+\\
\hline%----------------------------------------------------
\verb+(xi, yi, zi)+ &  the i-th vertex position\\
\hline%----------------------------------------------------
\end{tabular}
\vspace{.20in}

%%%%%%%%%%%%%%%%%%%%%%%%%%%%%%%%%%%%%%%%%%%%%%%%%%%%%
%The key word \verb+/PLVertex+ can be replaced with \verb+PLVertex+ 
%without the first letter ^^ /', though it is not recommended.
%%%%%%%%%%%%%%%%%%%%%%%%%%%%%%%%%%%%%%%%%%%%%%%%%%%%%

%------------------
\subsection{/Sphere}  
%------------------
\verb+/Sphere+ describes a sphere.
Its center locates at the origin.   
\vspace{.20in}

\begin{tabular}{|c|l|}
\hline%----------------------
Format & \verb+/Sphere  r+\\
\hline%----------------------
\verb+r+      & radius\\
\hline%----------------------
\end{tabular}


%------------------
\subsection{/SphereSeg}  
%------------------
\verb+/SphereSeg+ describes a sphere segment in polar and/or azimuthal angles.  
It corresponds to class G4Sphere of GEANT4.
\vspace{.20in}

\begin{tabular}{|c|l|}
\hline%---------------------------------------
Format & \verb+/SphereSeg  rmin rmax stheta dtheta sphi dphi+\\
\hline%---------------------------------------
\verb+rmin+  & minimum (inside)  radius \\
\hline%---------------------------------------
\verb+rmax+  & maximum (outside) radius \\
\hline%---------------------------------------
\verb+stheta+  & starting polar angle, $[0, \pi$] \\
\hline%---------------------------------------
\verb+dtheta+  &  extension of polar angle, \verb+dtheta+ = $[0, \pi]$\\
               &  and \verb+stheta+ + \verb+dtheta+ = $[0, \pi$] \\
\hline%---------------------------------------
\verb+sphi+  & starting azimuthal angle, $[-2\pi, +2\pi$] \\
\hline%---------------------------------------
\verb+dphi+  &  extension of azimuthal angle, \verb+dphi+ = $[0, 2\pi]$\\
             &  and \verb+sphi+ + \verb+dphi+ = $[-2\pi, +2\pi$] \\
\hline%---------------------------------------
\end{tabular}

%------------------
\subsection{/Torus}
%------------------
\verb+/Torus+ describes a torus or its segment in azimuthal angle
around the z axis.
The curbed tube to form the torus has 
a constant minimum inside radius (rmin) and a maximum outside radius (rmax).
The circle drawn by the central line of the tube has a radius rtor,
and center of this circle is the coordinate origin.
\verb+/Torus+ corresponds to class G4Torus of GEANT4.
\vspace{.20in}

\begin{tabular}{|c|l|}
\hline%---------------------------------------
Format & \verb+/Torus rmin rmax rtor sphi dphi+\\
\hline%---------------------------------------
\verb+rmin+  & minimum (inside)  radius of tube \\
\hline%---------------------------------------
\verb+rmax+  & maximum (outside) radius of tube \\
\hline%---------------------------------------
\verb+rtor+  & radius of circle drawn by the central line of the tube\\
\hline%---------------------------------------
\verb+sphi+ & starting azimuthal angle of the above circle, \\
            & \verb+sphi+ = $[-2\pi, +2\pi$]\\
\hline%---------------------------------------
\verb+dphi+ & extension of azimuthal angle of the above circle,\\
	    & \verb+dphi+ = $[0, 2\pi]$,\\
            &  \verb+sphi+ + \verb+dphi+ = $[-2\pi, +2\pi$] \\
\hline%---------------------------------------
\end{tabular}


%------------------
\subsection{/Trap}
%------------------
\verb+/Trap+ is a skewed version of \verb+/Trd+, i.e.,   
asymmetric pyramid with its upper part cut away.
Its top and bottom facets are asymmetric trapezoids.   
Its skew is expressed with direction of   
a line joining the centers of the top and bottom trapezoids.  
(Here we define a center of a trapezoid  by an intersection   
point of two lines passing through middle points of opponent edges.)\ \   
This line should pass through the origin.
The     top    trapezoid is on plane z = +dz,   
and the bottom trapezoid    on plane z = -dz.
Note that there are 11 parameters, but only 9 are really independent.
Meanings of some parameters are similar to those of \verb+/Parallelepiped+.
\verb+/Trap+ corresponds to class G4Trap of GEANT4.
\vspace{.20in}

\begin{tabular}{|c|l|}
\hline%---------------------------------------
Format & \verb+/Trap dz theta phi h1 bl1 tl1 alpha1+ \\
       & \verb+                   h2 bl2 tl2 alpha2+ \\
\hline%---------------------------------------
\verb+dz+     & half height of this shape along the z axis \\
\hline%---------------------------------------
\verb+theta+  & polar angle of the line expressing skew \\
\hline%---------------------------------------
\verb+phi+    & azimuthal angle of the line expressing skew \\
\hline%---------------------------------------
\verb+h1+     & half height of the bottom trapezoid along the y axis \\
\hline%---------------------------------------
\verb+bl1+    & half length along the x direction of the side at \\
              & minumum y of the bottom trapezoid\\
\hline%---------------------------------------
\verb+tl1+    & half length along the x direction of the side at \\
              & maximum y of the bottom trapezoid \\
\hline%---------------------------------------
\verb+alpha1+ & angle formed the y axis and a line joining middle points \\
              & of the two x-directional edges of the bottom trapezoid \\
\hline%---------------------------------------
\verb+h2+     & half height of the top trapezoid along the y axis \\
\hline%---------------------------------------
\verb+bl2+    & half length along the x direction of the side at \\
              & minimum y of the top trapezoid \\
\hline%---------------------------------------
\verb+tl2+    & half length along the x direction of the side at \\
              & maximum y of the top trapezoid \\
\hline%---------------------------------------
\verb+alpha2+ & angle formed the y axis and a line joining middle points \\
              & of the two x-directional edges of the bottom trapezoid\\
\hline%---------------------------------------
\end{tabular}


%------------------
\subsection{/Trd}
%------------------
\verb+/Trd+ describes a symmetric pyramid with its upper part cut away.
Its properties are:

\begin{enumerate}
\item The top and bottom facets are rectangles with, in general, different sizes.
\item Two edges of the top (or bottom) rectangles   
      are parallel to the x axis, and   
      the other two edges are parallel to the y axis.
\item The     top    rectangle is  on plane  z = +dz,  
      and the bottom rectangle     on plane  z = -dz.
\item Centers of the top and bottom rectangles are   
      (0, 0, +dz) and  (0, 0, -dz), respectively.
\end{enumerate}
\verb+/Trd+ corresponds to class G4Trd of GEANT4.
\vspace{.20in}

\begin{tabular}{|c|l|}
\hline%---------------------------------------
Format & \verb+/Trd  dx1 dx2 dy1 dy2 dz+\\
\hline%---------------------------------------
\verb+dx1+    & half length of x-parallel edges at z = -dz\\
\hline%---------------------------------------
\verb+dx2+    & half length of x-parallel edges at z = +dz\\
\hline%---------------------------------------
\verb+dy1+    & half length of y-parallel edges at z = -dz\\
\hline%---------------------------------------
\verb+dy2+    & half length of y-parallel edges at z = +dz\\
\hline%---------------------------------------
\verb+dz+     & half height along the z axis \\
\hline%---------------------------------------
\end{tabular}


%------------------
\subsection{/Tubs}
%------------------
\verb+/Tubs+ describes a tube or its segment in azimuthal angle  
with constant minimum (inside) and maximum (outside) radii.
Its height extends along the z axis.
The     top    facet is on plane z = +dz,   
and the bottom facet    on plane z = -dz.
Centers of the top and bottom facets are   
(0, 0, +dz) and  (0, 0, -dz), respectively.
\verb+/Tubs+ is equivalent to \verb+/Cons+    
with \verb+rmin1+ = \verb+rmin2+ and \verb+rmax1+ = \verb+rmax2+.
\verb+/Tubs+ corresponds to class G4Tubs of GEANT4.
\vspace{.20in}

\begin{tabular}{|c|l|}
\hline%---------------------------------------
Format & \verb+/Tubs rmin rmax dz sphi dphi+\\
\hline%---------------------------------------
\verb+rmin+  & minimum (inside)  radius \\
\hline%---------------------------------------
\verb+rmax+  & maximum (outside) radius \\
\hline%---------------------------------------
\verb+dz+    & half height along the z axis\\
\hline%---------------------------------------
\verb+sphi+  & starting azimuthal angle, $[-2\pi, +2\pi$] \\
\hline%---------------------------------------
\verb+dphi+  &  extension of azimuthal angle, \verb+dphi+ = $[0, 2\pi]$, \\
             &  \verb+sphi+ + \verb+dphi+ = $[-2\pi, +2\pi$] \\
\hline%---------------------------------------
\end{tabular}


%%%%%%%%%%%%%%%%%%%%%%%%%%%%%%%%%%%
\section{Formats of attributes}
%%%%%%%%%%%%%%%%%%%%%%%%%%%%%%%%%%%

This section describes the formats of available attributes.
Each attribute has an current value,  
which is assigned to described 3D primitives automatically.  

%---------------------
\subsection{/ColorRGB}  
%---------------------
\verb+/ColorRGB+ describes current color.  
For 3D shapes such as \verb+/Box+,  
it is used as a set of diffusion reflection coefficients.  
For polylines, it is regarded as their color directly.
By default, \verb+/ColorRGB 1.0 1.0 1.0+ is set implicitly. 
\vspace{.20in}

\begin{tabular}{|c|l|}
\hline%---------------------------------------
Format & \verb+/ColorRGB  R  G  B+\\
\hline%---------------------------------------
(R, G, B) & red, green, and blue components of color. \\
          & R, G, B = [0,1] \\
\hline%---------------------------------------
\end{tabular}


%----------------------------
\subsection{/FontName}  
%----------------------------
\verb+/FontName+ sets a current font used by 
\verb+/MarkText2D+ and \verb+/MarkText2DS+, and \verb+/Text2DS+. 
Its argument should coincide with one of a font name supported by 
a PostScript driver used for printing.
Greek letters are also available by setting the argument to "Symbol".
By default, \verb+/FontName Times-Roman+ is set implicitly. 
\vspace{.20in}

\begin{tabular}{|c|l|}
\hline%---------------------------------------
Format     & \verb+/FontName  fontname+\\
\hline%---------------------------------------
fontname   &  string to specify font name \\
           &  (e.g. Times-Roman, Courier, Symbol) \\
\hline%---------------------------------------
\end{tabular}



%-------------------------
\subsection{/ForceWireframe}  
%-------------------------
\verb+/ForceWireframe+ is a boolean flag, which describes current 
forcible wireframe mode.
Its value is either 0 or 1.
If it is set to 1, 3D shapes are visualized with 
wireframe style forcibly.
If it is set to 0, 3D shapes are visualized according to 
a drawing style given on the GUI panel of DAWN.
By default, \verb+/ForceWireframe 0+ is set implicitly. 
\vspace{.20in}

\begin{tabular}{|c|l|}
\hline%----------------------------------------
Format     & \verb+/ForceWireframe  flag+\\
\hline%----------------------------------------
\verb+flag+       & 1: forcible wireframe mode is made active\\
                  & 0: forcible wireframe mode is made inactive \\
\hline%----------------------------------------
\end{tabular}
\vspace{.20in}

\verb+/ForceWireframe+ is often used to make a shape look transparent. 
For example, the following lines described in the modeling block 
displays a green opaque sphere put in a red transparent box:
\begin{verbatim}
### green opaque sphere (inside)
/ColorRGB 0.0  1.0  0.0 
/ForceWireframe 0
/Sphere     1.0
### red transparent box (outside)
/ColorRGB 1.0  0.0  0.0 
/ForceWireframe 1
/Box 1.1  1.1 1.1
\end{verbatim}
\vspace{.20in}

\noindent
DAWN does not support a way of expressing 
half-transparent features of shapes.  


%------------------
\subsection{/Ndiv}  
%------------------
\verb+/Ndiv+ describes current accuracy in approximating a curved surface
with a set of polygons.
For \verb+/Column+, \verb+/Cons+, \verb+/PolyCone+, \verb+/PolyGon+,   
and \verb+/Tubs+,  
each of their side surfaces is expressed with \verb+/Ndiv+ small rectangles.  
For \verb+/Sphere+, its surface is expressed with  
$2\times$ \verb+/Ndiv+ $\times$ \verb+/Ndiv+ small rectangles or triangles.
By default, \verb+/Ndiv 24+ is set implicitly. 
\vspace{.20in}

\begin{tabular}{|c|l|}
\hline%----------------------------------------
Format     & \verb+/Ndiv  n+\\
\hline%----------------------------------------
\verb+n+   & accuracy of approximating curved surfaces, n $\geq 3$\\
\hline%----------------------------------------
\end{tabular}


%%%%%%%%%%%%%%%%%%%%%%%%%%%%%%%%%%%%%%%%%%%%%%%%%%
%%-------------------------
%\subsection{/Transparency}  
%%-------------------------
%\verb+/Transparency+ is a boolean flag to describe  
%transparency or opaqueness of 3D shapes.
%Its value is either 0 or 1.
%Note that DAWN draws transparent shapes with
%the forced wireframe style.  
%Therefore \verb+/Transparency+  is equivalent to 
%\verb+/Forcewireframe+.
%\vspace{.20in}
%
%\begin{tabular}{|c|l|}
%\hline%----------------------------------------
%Format     & \verb+/Transparency  flag+\\
%\hline%----------------------------------------
%\verb+flag+       & 1: transparent\\
%                  & 0: opaque \\
%\hline%----------------------------------------
%\end{tabular}
%%%%%%%%%%%%%%%%%%%%%%%%%%%%%%%%%%%%%%%%%%%%%%%%%%%



%%%%%%%%%%%%%%%%%%%%%%%%%%%%%%%%%%%
\section{Formats of markers}  
%%%%%%%%%%%%%%%%%%%%%%%%%%%%%%%%%%%

Visible markers can be set to arbitrary 3D positions.  
They correspond to classes inherited from G4VMarker in GEANT4. 
The ending string ^^ ^^ 2D'' or ^^ ^^ 2DS''  of each marker name,
e.g. \mbox{\verb+/MarkSquare2D+} and \mbox{\verb+/MarkSquare2DS+},
means that every marker always shows its front face towards the view point,
and so, for example, a marker \verb+/MarkSquare2D+  always looks
as right square.
The marked 3D positions are expressed with the current body coordinates.  
Colors are decided by the current color attribute.
As for marker size, there are some points to be well understood in use:
\begin{enumerate}
\item A unit of marker size is either the real 3D-world unit
      or the 2D unit on screen:
	\begin{enumerate}      
	\item Size of a marker whose name ends with character ^^ S', e.g.\\
	      \verb+/MarkCircle2DS+, 
	      is described with the 2D unit on screen. 
	\item  Size of a marker whose name does not end with character ^^ S', 
	       e.g. \verb+/MarkCircle2D+ is described with the 3D unit.
	\end{enumerate}
\item The 2D unit is recognized as ^^ ^^ pixel'' on computer displays
      and ^^ ^^ pt'' \mbox{(1[pt]= (25.4/72)mm)} on printed out papers.
\item For \verb+/MarkText2D+ and \verb+/MarkText2DS+, their marker  sizes
      mean font sizes. 
\end{enumerate}
Note that in GEANT4 sizes of markers are given as diameters, 
while in g4.prim-format data they are given as radii.  

An important feature of markers is that they are always
drawn at the end of visualization, such that they can always
become visible. In other words,
hidden surface elimination is not applied to markers.



%----------------------------
\subsection{/MarkCircle2D}
%----------------------------
\verb+/MarkCircle2D+ describes a marker of a two-dimensional circle
put in the 3D world.
\verb+/MarkCircle2D+ corresponds to class G4Circle of GEANT4 with
its size given as world-coordinate size.
\vspace{.20in}

\begin{tabular}{|c|l|}
\hline%---------------------------------------
Format     & \verb+/MarkCircle2D  x y z  r+\\
\hline%---------------------------------------
(x, y, z)  &  marked 3D position\\
\hline%---------------------------------------
r          &  marker size (3D unit): radius of the circle  \\
\hline%---------------------------------------
\end{tabular}


%----------------------------
\subsection{/MarkCircle2DS}
%----------------------------
\verb+/MarkCircle2DS+ describes a marker of a two-dimensional circle
put in the 3D world.
\verb+/MarkCircle2DS+ corresponds to class G4Circle of GEANT4 with
its size given as screen size.
\vspace{.20in}

\begin{tabular}{|c|l|}
\hline%---------------------------------------
Format     & \verb+/MarkCircle2DS  x y z  r+\\
\hline%---------------------------------------
(x, y, z)  &  marked 3D position\\
\hline%---------------------------------------
r          &  marker size (2D unit): radius of the circle     \\
\hline%---------------------------------------
\end{tabular}

%----------------------------
\subsection{/MarkSquare2D}  
%----------------------------
\verb+/MarkSquare2D+ describes a marker of a two-dimensional right square  
put in the 3D world.
\verb+/MarkSquare2D+ corresponds to class G4Square of GEANT4 with
its size given as world-coordinate size.
\vspace{.20in}

\begin{tabular}{|c|l|}
\hline%---------------------------------------
Format     & \verb+/MarkSquare2D  x y z  dx+\\
\hline%---------------------------------------
(x, y, z)  &  marked 3D position  \\
\hline%---------------------------------------
dx         &  marker size (3D unit): half edge length of the square \\
\hline%---------------------------------------
\end{tabular}


%----------------------------
\subsection{/MarkSquare2DS}  
%----------------------------
\verb+/MarkSquare2D+ describes a marker of a two-dimensional right square  
put in the 3D world.
\verb+/MarkSquare2DS+ corresponds to class G4Square of GEANT4 with
its size given as screen size.
\vspace{.20in}

\begin{tabular}{|c|l|}
\hline%---------------------------------------
Format     & \verb+/MarkSquare2DS  x y z  dx+\\
\hline%---------------------------------------
(x, y, z)  &  marked 3D position  \\
\hline%---------------------------------------
dx         &  marker size (2D unit): half edge length of the square \\
\hline%---------------------------------------
\end{tabular}


%----------------------------
\subsection{/MarkText2D}  
%----------------------------
\verb+/MarkText2D+ displays a text (string) to a marked 3D position. 
The marked position is the left end of the text.
\verb+/MarkText2D+ corresponds to class G4Text of GEANT4 with
its size given as world-coordinate size.
\vspace{.20in}

\begin{tabular}{|c|l|}
\hline%---------------------------------------
Format     & \verb+/MarkText2D  x y z size x_offset y_offset string+\\
\hline%---------------------------------------
(x, y, z)  &  marked 3D position (left end of string) \\
\hline%---------------------------------------
size       &  font size (3D unit) \\
\hline%---------------------------------------
x\_offset   &  horizontal offset (3D unit) \\
\hline%---------------------------------------
y\_offset   &  vertical   offset (3D unit) \\
\hline%---------------------------------------
string     &  string to be displayed (Do not quote with ^^ ^^ ''.)\\
\hline%---------------------------------------
\end{tabular}
\vspace{.20in}


%----------------------------
\subsection{/MarkText2DS}  
%----------------------------
\verb+/MarkText2DS+ displays a text (string) to a marked 3D position. 
The marked position is the left end of the text.
\verb+/MarkText2DS+ corresponds to class G4Text of GEANT4 with
its size given as screen size.
\vspace{.20in}

\begin{tabular}{|c|l|}
\hline%---------------------------------------
Format     & \verb+/MarkText2DS  x y z size x_offset y_offset string+\\
\hline%---------------------------------------
(x, y, z)  &  marked 3D position  \\
\hline%---------------------------------------
size       &  font size  (2D unit, i.e.~pt) \\
\hline%---------------------------------------
x\_offset   &  horizontal offset (2D units, i.e.~pt) \\
\hline%---------------------------------------
y\_offset   &  vertical   offset (2D units, i.e.~pt) \\
\hline%---------------------------------------
string     &  string to be displayed (Do not quote with ^^ ^^ ''.)\\
\hline%---------------------------------------
\end{tabular}
\vspace{.20in}

\noindent
For example, suppose that we want to display a string 
"Fukui Renderer DAWN" with the following attributes:

\begin{itemize}
\item   position: (1,2,3) in the current body coordinate
\item   color   : red  
\item   font name  : Courier  
\item   font size  : 12      (2D unit)
\item   horizontal offset: 6 (2D unit) 
\item   vertical offset:   3 (2D unit) 
\end{itemize}

\noindent
The following lines described in the modeling block 
displays this string on the screen.
\begin{verbatim}
/ColorRGB  1 0 0  
/FontName Courier 
/MarkText2DS 1 2 3 12 6 3 Fukui Renderer DAWN
\end{verbatim}
\vspace{.20in}

\noindent
Note that all arguments beginning with the 7-th argument, 
i.e., ^^ ^^ Fukui'' in the above example, 
are grouped and treated as one string.



%%%%%%%%%%%%%%%%%%%%%%%%%%%%%%%%%%%%%%%%%%%%%%
\section{Drawing Texts directly on 2D screen} 
%%%%%%%%%%%%%%%%%%%%%%%%%%%%%%%%%%%%%%%%%%%%%%

%----------------------------
\subsection{/Text2DS} 
%----------------------------

\verb+/Text2DS+ displays a text (string) directly 
on the two-dimensional space on screen.
No corresponding classes exist in GEANT4.
Note that its position, i.e., left end of the string,
is described with 2D position vector described 
in units of ^^ ^^ mm'',
and its origin is the left bottom of the screen.
Font size should be given in units of ^^ ^^ pt''.
\verb+/Text2DS+ is convenient to add title strings etc 
to drawn figures afterwards.
\vspace{.20in}

\begin{tabular}{|c|l|}
\hline%---------------------------------------
Format        & \verb+/Text2DS  x_mm y_mm  size_pt string+\\
\hline%---------------------------------------
(x\_mm, y\_mm)  &  Left end of string on screen. \\
		&  Unit is mm. Origin (0,0) is the left-bottom.\\
\hline%---------------------------------------
size       &  font size (in units of pt) \\
\hline%---------------------------------------
string     &  string to be displayed \\
\hline%---------------------------------------
\end{tabular}
\vspace{.20in}

\noindent
For example, the following lines described in the modeling block 
displays a large title "Supported Primitives 1" at the bottom center of 
the screen:  
\vspace{.20in}

\begin{verbatim}
### black color 
/ColorRGB 0 0 0 
### font name  
/FontName Times-Roman 
### 40-pixel string at (30, 40) on screen 
/Text2DS 30 40 40 Supported Primitives 1
\end{verbatim}


%%%%%%%%%%%%%%%%%%%%%%%%%%%%%%%%
\section{Sample g4.prim-format data files}   
%%%%%%%%%%%%%%%%%%%%%%%%%%%%%%%%
Many viewable sample g4.prim-format data files
are included in the DAWN package.  
For example, see files \verb+primitives.prim+ 
and \verb+primitives2.prim+.



%%%%%%%%%%%%%%%%%%%%%%%%%%%%%%%%%%%%%%%%%%%%%%%%%%%%%%%%%%%%%%%%%%%%%%
%%%%%%%%%%%%%%%%%%%%%%%%%%%%%%%%%
%\section{Relation to an old g4.prim format}   
%%%%%%%%%%%%%%%%%%%%%%%%%%%%%%%%%
%
%The g4.prim format version 2.0 is adopted by DAWN version 3.00a. 
%There is an old g4.prim format which is readable by DAWN upto version 2.XX, 
%and is not readable by DAWN version 3.XX.
%
%If you want to visualize an old g4.prim-format file with DAWN version 3.XX, 
%edit the file as follows:
%
%\begin{enumerate} 
%\item Add a heading comment line beginning with ^^ ^^ \verb+##+'' 
%\item Replace each line of \verb+/MarkCircle2D x y z r R G B+ 
%with the following two lines:
%\begin{verbatim}
%/ColorRGB  R G B
%/MarkCircle2D x y z r  
%\end{verbatim}
%\item Do the same thing for \verb+/MarkSquare2D x y z r R G B+. 
%\end{enumerate}
%%%%%%%%%%%%%%%%%%%%%%%%%%%%%%%%%%%%%%%%%%%%%%%%%%%%%%%%%%%%%%%%%%%%%%

\end{document}

